\chapter{Methodology}\label{sec:method}
\wannier\ computes maximally-localised Wannier functions (MLWF)
following the method of Marzari and Vanderbilt
(MV)~\cite{marzari-prb97}.  For entangled energy bands, the method of
Souza, Marzari and Vanderbilt (SMV)~\cite{souza-prb01} is used. We
introduce briefly the methods and key definitions here, but full
details can be found in the original papers and in
Ref.~\cite{mostofi-cpc08}.

First-principles codes typically solve the electronic structure of
periodic materials in terms of Bloch states, $\psi_{n{\bf k}}$. 
These extended states are characterised by a band index $n$ and crystal
momentum ${\bf k}$. An alternative representation can be given in terms
of spatially localised functions known as Wannier functions (WF). The WF 
centred on a lattice site ${\bf R}$, $w_{n{\bf R}}({\bf r})$, 
is written in terms of the set of Bloch states as
\begin{equation}
w_{n{\bf R}}({\bf r})=\frac{V}{(2\pi)^3}\int_{\mathrm{BZ}}
\left[\sum_{m} U^{({\bf k})}_{mn} \psi_{m{\bf k}}({\bf
    r})\right]e^{-\mathrm{i}{\bf k}.{\bf R}} \:\mathrm{d}{\bf k} \ , 
\end{equation}
where $V$ is the unit cell volume, the integral is over the Brillouin
zone (BZ), and $\Uk$ is a unitary matrix that mixes the Bloch 
states at each ${\bf k}$. $\Uk$ is not uniquely defined and different
choices will lead to WF with varying spatial localisations. We define
the spread $\Omega$ of the WF as 
\begin{equation}
\Omega=\sum_n \left[\langle w_{n{\bf 0}}({\bf r})| r^2 | w_{n{\bf
      0}}({\bf r}) \rangle - | \langle w_{n{\bf 0}}({\bf r})| {\bf r}
      | w_{n{\bf 0}}({\bf r}) \rangle |^2 \right].
\end{equation}
The total spread can be decomposed into a gauge invariant term
$\Omega_{\rm I}$ plus a term ${\tilde \Omega}$ that is dependant on the gauge
choice $\Uk$. ${\tilde \Omega}$ can
be further divided into terms diagonal and off-diagonal in the WF basis,
$\Omega_{\rm D}$ and $\Omega_{\rm OD}$,
\begin{equation}
\Omega=\Omega_{\rm I}+{\tilde \Omega}=\Omega_{\rm I}+\Omega_{\rm
  D}+\Omega_{\rm OD} 
\end{equation}
where
\begin{equation}
\Omega_{{\rm I}}=\sum_n \left[\langle w_{n{\bf 0}}({\bf r})| r^2 | w_{n{\bf
      0}}({\bf r}) \rangle - \sum_{{\bf R}m} \left| \langle w_{n{\bf
      R}}({\bf r})| {\bf r} | w_{n{\bf 0}}({\bf r}) \rangle \right| ^2
      \right] 
\end{equation}
\begin{equation}
\Omega_{\rm D}=\sum_n \sum_{{\bf R}\neq{\bf 0}} |\langle w_{n{\bf
    R}}({\bf r})| {\bf r} | w_{n{\bf 0}}({\bf r}) \rangle|^2 
\end{equation}
\begin{equation}
\Omega_{\rm OD}=\sum_{m\neq n} \sum_{{\bf R}} |\langle w_{m{\bf R}}({\bf
  r})| {\bf r} | w_{n{\bf 0}}({\bf r}) \rangle |^2 
\end{equation}
The MV method minimises the gauge dependent spread $\tilde{\Omega}$
with respect the set of $\Uk$ to obtain MLWF.

\wannier\ requires two ingredients from an initial electronic
structure calculation. 
\begin{enumerate}
\item The overlaps between the cell periodic part of the Bloch states
  $|u_{n{\bf k}}\rangle$  
\begin{equation}
M_{mn}^{(\bf{k,b})}=\langle u_{m{\bf k}}|u_{n{\bf k}+{\bf b}}\rangle,
\end{equation}
where the vectors ${\bf b}$, which connect a given k-point with its
neighbours, are determined by \wannier\ according to the prescription
outlined in Ref.~\cite{marzari-prb97}.
\item As a starting guess the projection of the Bloch states
  $|\psi_{n\bf{k}}\rangle$ onto trial localised orbitals $|g_{n}\rangle$ 
\begin{equation}
A_{mn}^{(\bf{k})}=\langle \psi_{m{\bf k}}|g_{n}\rangle,
\end{equation}
\end{enumerate}
Note that $\Mkb$, $\Ak$ and $\Uk$ are all small, $N \times N$
matrices\footnote{Technically, this is true for the case of an
  isolated group of $N$ bands from which we obtain $N$ MLWF. When
  using the disentanglement procedure of Ref.~\cite{souza-prb01},
  $\Ak$, for example, is a rectangular matrix. See
  Section~\ref{sec:disentangle}.}  that are independent of the basis
set used to obtain the original Bloch states.

To date, \wannier\ has been used in combination with electronic codes
based on plane-waves and pseudopotentials (norm-conserving and
ultrasoft~\cite{vanderbilt-prb90}) as well as mixed basis set techniques such as
FLAPW~\cite{posternak-prb02}.

\section{Entangled Energy Bands}\label{sec:disentangle}
The above description is sufficient to obtain MLWF for an isolated set
of bands, such as the valence states in an insulator. In order to
obtain MLWF for entangled energy bands we use the ``disentanglement''
procedure introduced in Ref.~\cite{souza-prb01}.

We define an energy window (the ``outer window''). At a given
k-point $\bf{k}$, $N^{({\bf k})}_{{\rm win}}$ states lie within this
energy window. We obtain a set of $N$ Bloch states by
performing a unitary transformation amongst the Bloch states which
fall within the energy window at each k-point: 
 \begin{equation}
| u_{n{\bf k}}^{{\rm opt}}\rangle = \sum_{m\in N^{({\bf k})}_{{\rm win}}}
U^{{\rm dis}({\bf k})}_{mn} | u_{m{\bf k}}\rangle
\end{equation}
where $\bf{U}^{{\rm dis}({\bf k})}$ is a rectangular $N \times N^{({\bf k})}_{{\rm win}}$
 matrix\footnote{As $\bf{U}^{{\rm dis}({\bf k})}$ is a rectangular
 matrix this is a unitary operation in the sense that $(\bf{U}^{{\rm
 dis}({\bf k})})^{\dagger}\bf{U}^{{\rm dis}({\bf k})}=\bf{1}$.}. The
 set of $\bf{U}^{{\rm dis}({\bf k})}$ are obtained by minimising 
 the gauge invariant spread $\Omega_{{\rm I}}$ within the outer energy
 window. The MV procedure can then be used to minimise $\tilde{\Omega}$
 and hence obtain MLWF for this optimal subspace.

It should be noted that the energy bands of this optimal subspace may
not correspond to any of the original energy bands (due to mixing
between states). In order to preserve exactly the properties of a
system in a given energy range (e.g., around the Fermi level) we
introduce a second  energy window. States lying within this inner, or
``frozen'', energy window are included unchanged in the optimal
subspace.
